\section{Introduction}

Bitcoin~\cite{bitcoin} has emerged as the most successful \emph{cryptocurrency}
in history. Its innovative technology has introduced \emph{blockchains}, a
cryptographically \emph{authenticated data
structure}~\cite{miller2014authenticated} in which monetary and other
\emph{transactions} are organized into \emph{blocks} which are then put in
order. To achieve \emph{consensus} about their order, a
\emph{proof-of-work}~\cite{C:DwoNao92} is included in each block. Transactions
moving value within such blockchains have been proven to be secure in that
consensus is eventually achieved \cite{EC:GarKiaLeo15}\cite{C:GarKiaLeo17},
solving the long-standing \emph{anonymous byzantine agreement} problem in the
literature of distributed computing~\cite{miller2014anonymous}.

Ethereum~\cite{buterin} extends Bitcoin's functionality and introduces the
ability to write arbitrary code to be run on the blockchain through
Turing-complete \emph{smart contracts} written in programming languages such as
Solidity. These smart contracts execute autonomously on blockchains.
While the Ethereum Virtual Machine~\cite{wood} executing such contracts allows
for Turing-completeness, the smart contracts are generally confined to access
data only within the blockchain itself, such as previous transactions and
blocks. Access to external world data requires some third party or group of
third parties to vouch for the validity of the data~\cite{CCS:ZCCJS16},
essentially mandating the need for the trusted third party or trusted group of
third parties.

Sidechains~\cite{sidechains} are a mechanism for cross-chain communication in
blockchains. They allow the smart contracts on one blockchain to receive and
react to \textit{events} that take place on another blockchain without the need
of a trusted third party.

\noindent\textbf{Our contributions. } In this paper, we introduce the first
trustless construction for proof-of-work sidechains. We describe how to build
sidechains that support \emph{two-way pegging} and in which an asset can be
moved from one blockchain to another while retaining its nature. We illustrate
how to move assets by providing a high-level construction for Turing-complete
blockchains such as Ethereum in Solidity, and we prove our construction is
secure under the assumption that the interoperating blockchains are secure
individually. To our knowledge, we are the first to provide such a construction
in full and prove its security. Our construction works across a broad range of
blockchains requiring only two underlying properties. First, that the
\emph{source} blockchain is a proof-of-work blockchain supporting
Non-Interactive Proofs of Proof-of-Work (NIPoPoWs), a cryptographic primitive
which allows constructing succinct proofs \emph{about} events which occur in a
proof-of-work blockchain and which was recently introduced by Miller et
al.~\cite{nipopows}. Support for NIPoPoWs can be easily introduced to practically
any work-based cryptocurrency which uses a localized hash function to achieve
proof-of-work, such as SHA256 as used by Bitcoin, Bitcoin Cash, Litecoin or
ZCash. Second, that the \emph{target} blockchain is able to validate such proofs
by, for instance, being Turing-complete, such as, e.g., Ethereum or Ethereum
Classic. Any blockchain supporting advanced smart contracts is sufficient.

We provide a smart contract skeleton for how to build such pegs.

\noindent\textbf{Related work. }
Sidechains were introduced as a Bitcoin upgrade mechanism by Back et
al.~\cite{sidechains}. The upgrade mechanism proposed introducing new
\emph{child} blockchains with a new genesis block, which implement a new version
of the protocol. The asset handled by such sidechains is the same as the asset
of the mainchain and can be moved between the mainchain and the sidechain
through \emph{2-way pegging}. The \emph{firewall} property was also introduced
in their work, in that, a macroeconomic failure in the sidechain, such as a
security problem which could allow the violation of money supply policies, will
not impact the mainchain. No security definitions or construction of the
protocol were provided in this paper. However, their paper hints at the need for
``\emph{secure SPV proofs}'' (Appendix B) in future work (which we implement
here). We use the word \emph{sidechain} in a more general notion than in the
work by Back et al. Our sidechains allow communication between \emph{stand
alone} blockchains. In our work, a blockchain is a sidechain of another chain if
it can detect and react to events on that chain.

\emph{Polkadot}~\cite{wood2016polkadot}, \emph{Tendermint},
\emph{Cosmos}~\cite{buchman2016tendermint}, \emph{Liquid} and \emph{Interledger}
also build cross-chain transfers. Their validation relies on a trusted
committees or federations or is left unspecified. \emph{Drivechains} are a
trustless sidechain proposal, but require miners of both the chains to be aware
of both networks. In our scheme, miners remain agnostic to the existence of
other chains and connect only to one network. \emph{BTCRelay} is a trustless
mechanism that allows relaying information one-way from Bitcoin to Ethereum. The
miners can remain connected to their respective chain network only. Unlike our
proposal, BTCRelay requires the retransmission of the entirety of the source
blockchain headers into the target blockchain, data linear in the side of the
source blockchain. Our proposal instead only requires submitting parameters
logarithmic in size of the source blockchain. This is inherited by the
\emph{succinctness} property of the NIPoPoW scheme. None of the aforementioned
constructions include proofs of security.
