\section{Introduction}

Bitcoin~\cite{bitcoin} is the first and most successful \emph{cryptocurrency} to
date. Its core protocol introduced the concept of a  \emph{blockchain}, a type
of cryptographic consensus protocol in which \emph{transactions} are organized
into \emph{blocks} which are then put in a mutually agreed sequence despite the
presence of adversarial nodes. Consensus is achieved via  a
\emph{proof-of-work}~\cite{C:DwoNao92} which is the precondition for a block to
be valid.  Transactions moving value within such blockchains have been proven to
be secure in that consensus is eventually achieved, cf.
\cite{EC:GarKiaLeo15,EC:PasSeeShe17,C:GarKiaLeo17}, thus providing a way for
reaching consensus in a setting   where neither reliable point-to-point channels
exists nor a public-key infrastructure.

Ethereum~\cite{buterin} extends Bitcoin's functionality introducing the ability
to write arbitrary  Turing-complete \emph{smart contracts} in programming
languages such as Solidity running on top of the Ethereum Virtual
Machine~\cite{wood}. These contracts execute autonomously. The smart contracts
are confined to access data only within the blockchain itself, such as previous
transactions and blocks. Access to external world data requires some trusted
third party or group of third parties to vouch for the validity of the
data~\cite{CCS:ZCCJS16}.

Sidechains~\cite{sidechains} are a mechanism for cross-chain communication in
blockchains. They allow the smart contracts on one blockchain to receive and
react to \textit{events} that take place on another blockchain without the need
of a trusted third party. Despite the widely agreed usefulness of the primitive
there exist no constructions that are decentralised and efficient at the same
time.

\noindent\textbf{Our contributions. } In this paper, we introduce the first
trustless construction for proof-of-work sidechains. We describe how to build
generic communication between blockchains. As one application, we give the
construction of a \emph{two-way pegged} asset which can be moved from one
blockchain to another while retaining its nature. We provide a high-level
construction in Solidity. Our construction works across a broad range of
blockchains requiring only two underlying properties. First, that the
\emph{source} blockchain is a proof-of-work blockchain supporting
Non-Interactive Proofs of Proof-of-Work (NIPoPoWs), a cryptographic primitive
which allows constructing succinct proofs \emph{about} events which occur in a
proof-of-work blockchain and which was recently introduced in~\cite{nipopows}.
Support for NIPoPoWs can be introduced to practically any
work-based cryptocurrency such as Bitcoin, Ethereum, Bitcoin Cash, Litecoin or
Monero without a hard or soft fork. Second, that the \emph{target} blockchain is
able to validate such proofs by, for instance, being Turing-complete, such as,
e.g., Ethereum or Ethereum Classic. Any blockchain supporting advanced smart
contracts is sufficient. In the appendix, we give a formal proof of security of
our construction via reduction to NiPoPoW security under the assumption that the
interoperating blockchains are secure individually. To our knowledge, we are the
first to provide such a construction in full and prove its security.

\noindent\textbf{Related work. }
Sidechains were introduced as a Bitcoin upgrade mechanism by Back et
al.~\cite{sidechains}. They proposed introducing a new \emph{child} blockchain
which implements a new protocol version, with which assets are \emph{2-way
pegged}. The \emph{firewall} property was articulated. No security definitions
nor a complete construction of the protocol were given. Their paper hints at the
need for ``\emph{efficient SPV proofs}'' (Appendix B) in future work, which we
implemented here. We use the term \emph{sidechains} in a more general notion
than in their work. Our sidechains allow communication between \emph{stand
alone} blockchains and also convey \emph{any} information, not just transfers of
value. In our work, a blockchain is a sidechain of another chain if it can react
to events on that chain, and so the relationship can be symmetric.

\emph{Polkadot}~\cite{wood2016polkadot}, \emph{Tendermint},
\emph{Cosmos}~\cite{buchman2016tendermint}, \emph{Liquid} and
\emph{Interledger}~\cite{interledger} also build cross-chain transfers. Their
validation relies on a trusted committees, federations or is left unspecified.
\emph{Drivechains} are a sidechain proposal which requires miners of both chains
to be aware of both networks. In our scheme, miners remain agnostic to the
existence of other chains and connect only to one network. \emph{BTCRelay} is a
trustless mechanism relaying information one-way from Bitcoin to Ethereum, in
which miners are connected to their network only. BTCRelay requires the
transmission of the entirety of the source blockchain headers into the target
blockchain. Our proposal only requires data logarithmic in size of the source
blockchain. This stems from the \emph{succinctness} property of the NIPoPoW
scheme. None of the aforementioned constructions include proofs of security.
Other related work includes Plasma~\cite{poon2017plasma},
XCLAIM~\cite{zamyatinxclaim}, PeaceRelay, COMIT~\cite{comit}, and
NOCUST~\cite{khalil2018nocust} and Dogethereum.
