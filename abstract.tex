\begin{abstract}
During the last decade, the blockchain space has exploded with a plethora of new
cryptocurrencies, covering a wide array of different features, performance and
security characteristics.  Nevertheless,  each of these coins functions in a
stand-alone manner, independently.  Sidechains have been envisioned as a
mechanism to allow blockchains to communicate with one   another and, among
other applications,  allow the transfer of value from one chain to another, but
so far  there have been no decentralized constructions.  In this paper, we put
forth the first sidechains construction that allows communication between
proof-of-work blockchains without trusted intermediaries. Our construction is
generic in that it allows the passing of any information between blockchains.
Using this construction, two blockchains can be connected in a
``two-way peg'' in which an asset can be transferred from one chain to
another and back. We pinpoint the features needed for two chains to communicate:
On the source side, a proof-of-work blockchain that has been {\em interlinked},
potentially with a velvet fork; on the destination side, a blockchain with smart
contract support. We put forth the smart contracts needed to implement these
sidechains and explain them in detail.
\ifshort\else
We model our construction mathematically
and give a formal proof of security.
\fi
In the heart of our construction, we use a
recently introduced cryptographic primitive, Non-Interactive Proofs of
Proof-of-Work (NIPoPoWs).
\ifshort\else
Our security proof uses a standard computational
reduction from our new proof-of-work sidechains protocol to the security of
NIPoPoWs, which has, in turn, been shown to be secure in previous work. Our
working assumption is honest majority in each of the communicating chains.
\fi
\end{abstract}
