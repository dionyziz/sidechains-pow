\section{Applications and evaluation}

We now describe some of the possible applications of sidechains beyond the
two-way peg.

\noindent
\textbf{Remote ICOs. } A group wants to fundraise by performing an
ICO~\cite{ico} on a certain feature-rich \emph{target} blockchain such as
Ethereum. Investors who wish to purchase tokens are holding their cryptocurrency
they wish to pay with on a different \emph{source} blockchain with lower
volatility such as Bitcoin. In this context, instead of trading Bitcoin for
Ethereum, the investors can directly pay in Bitcoin and receive their ICO tokens
on the Ethereum blockchain. Furthermore, the value of tokens can be denominated
in Bitcoin.

The course of action is as follows: Initially, the fundraising entity creates a
\emph{fund raiser} address on the \emph{source blockchain}, in our example the
Bitcoin blockchain. The fundraising entity also creates a \emph{token generator}
smart contract on the \emph{target blockchain}, in our example the Ethereum
blockchain. The \emph{token generator} contract is initialized with the source
blockchain genesis hash and the \emph{fund raiser} address. The \emph{token
generator} contract contains a \emph{withdraw} method which allows receiving the
ICO tokens. The fundraising entity then advertises both the address and the
smart contracts and allows investors to inspect them. An investor subsequently
deposits bitcoin into the \emph{fund raiser} address. The investor subsequently
generates a proof-of-proof $\pi$ for the deposit and submits it to the
\emph{withdraw} method of the \emph{token generator} function. Upon verification
of $\pi$, the investor receives their ICO tokens. This application is a special
case of a \emph{one-way peg} with some extra rules such as a limited timeframe
or a potentially changing price.

\noindent
\textbf{Atomic swaps. } In this application, two participants wish to exchange
coins on two different chains, for example Ethereum for Ethereum Classic. Assume
Alice wishes to pay Bob in Ethereum and Bob wishes to pay Alice in Ethereum
Classic. They wish to do this while distrusting each other and without a
centralized exchange. While atomic swaps are already possible using hash
locks~\cite{atomicswaps,herlihy2018atomic}, our construction is an alternative
way of achieving them. In order to achieve it, one contract is created, on the
\emph{target} blockchain, in our example Ethereum, and is made aware of the
remote genesis block as well as address on the \emph{source} blockchain, in our
case Alice's Ethereum Classic address. Then, Alice deposits ether into the
\emph{deposit} function of the smart contract that sits on the \emph{source}
blockchain. This ether remains locked for a designated period of time. In the
meantime, Bob pays Alice in Ethereum Classic by creating a regular transaction
on the Ethereum Classic blockchain. He then generates a proof-of-proof $\pi$
that this transaction took place and is paying into Alice's account. By
submitting $\pi$ into the \emph{withdraw} function of the smart contract that
sits on the \emph{source} blockchain, Bob is able to receive his ether payment,
contingent on the fact that he has paid Alice. In case Bob aborts the protocol,
Alice is able to withdraw her money from the \emph{source} blockchain smart
contract by calling an \emph{abort} function, which is only callable as soon as
a sufficient number of blocks have passed. As long as Bob cooperates within a
timely manner, Alice is unable to call \emph{abort} due to the timelock, and
hence the protocol is atomic.

\noindent
\textbf{Firewalling complex blockchains. } The two most popular cryptocurrencies
are currently Bitcoin and Ethereum. The popularity of Bitcoin stems from the
fact that it has maintained stable behavior and can be relied on for security.
Their features are limited and their codebase is updated in a careful manner. On
the other hand, Ethereum innovates quickly and is Turing-complete. Hence,
Bitcoin is often used as a \emph{savings accounts} platform because it is safe,
while Ethereum is used as a \emph{checking accounts} platform because it is
feature rich. The problem stems from the fact that exchanges between these two
chains are constantly needed and are prone to volatility and market fees.
However, with sidechains, it is possible to get the best of both worlds. To
solve this problem in a new coin, a blockchain system consisting of \emph{two
blockchains} is created. The Bitcoin-like blockchain, or \emph{settlement
layer}, contains limited features and a codebase which is easily auditable. The
Ethereum-like blockchain, or \emph{computation layer}, is feature-rich and its
codebase is larger but harder to audit. The \emph{settlement layer} then issues
currency which can be transferred to the \emph{computation layer} (which does
not have its own currency) and back using two-way pegged sidechains, without
paying market fees or being subject to exchange volatility. In case a
catastrophic security issue occurs in the \emph{computation layer}, the
\emph{settlement layer} remains macroeconomically protected.

\noindent
\textbf{Scaling with sidechains. } Scalability is one of the most prominent
problems in blockchain space~\cite{bitcoinng,sompolinsky2013accelerating}.
Sidechains are a trivial way to \emph{shard} a blockchain system into multiple
blockchains, achieving higher transaction throughput. This method can be
combined with existing sharding techniques such as Bitcoin-NG~\cite{bitcoinng}
applied independently on each blockchain. Similar to the example above, a system
which wishes to achieve these scalability gains would have one baseline
blockchain designated as their \emph{settlement layer}. In this construction,
the settlement layer is the only blockchain which issues currency. In addition,
a separate blockchain is created on an as-needed basis, which does not have its
own currency, but can exchange coins with the parent \emph{settlement layer}
using two-way pegged sidechains. The creation of children blockchains can even
be user-initiated. If transactions occur intra-chain, they do not need to touch
the settlement layer. Inter-chain transactions between two different children
chains can occur by first moving coins from the source blockchain to the
settlement layer with one transaction and subsequently from the settlement layer
to the target blockchain with another. In order for this system to successfully
scale, most transactions need to be confined within a single blockchain. This
can be achieved if, for instance, each blockchain created is specific to a
particular industry. The suppliers and consumers of an industry can then
interact with each other within their designated blockchain and only need to
move their capital to the settlement layer in order to interact with suppliers
or consumers across industries.

\noindent
\textbf{Legacy interoperability. } A fundamental problem of blockchain systems
is that they are difficult to operate within the regulatory framework of
governments. This makes it difficult to move capital between legacy banking and
blockchains, as they stand at two extremes: Legacy banking is heavily regulated
to control every transaction, while blockchains, at least at the transaction
level, can be used despite regulation that may exist, which can remain
unenforceable, especially for anonymous users. Hence, banks are reluctant to
interoperate with blockchains. An intermediate solution can be reached with
sidechains. A parent blockchain can be created which is completely
decentralized. Subsequently, each bank, regulator, or government can operate a
private or semi-private blockchain~\cite{danezis2015centrally} as a sidechain
which can interact via two-way pegs with the parent decentralized chain. The
regulators can then apply the laws they desire on both intra-chain transactions
within the private blockchain as well as the cross-chain transfers between their
private blockchain and the decentralized parent blockchain. Such regulation
could, for instance, allow only white-listed entities to interact with the
system, or require a bank or government to sign off every transaction using
centrally held private keys. However, regulation concerning the decentralized
parent blockchain would remain unenforceable at the transaction level. As such,
moving assets from the decentralized blockchain into the regulated blockchain
would happen on a voluntary basis if a party deemed that it wishes to encumber
itself with a particular legal and regulatory framework.

\noindent
\textbf{Upgradability. } Sidechains also allow for upgradability, taking the
place of difficult to manage hard forks. After all, this was their original
purpose when they were envisioned~\cite{sidechains}. In the upgradability
application, a child blockchain is created to provide supplemental features to
an existing parent blockchain. Using two-way pegs, the two blockchains can
transfer assets between them on a voluntary basis, without requiring users to
follow a hard fork or soft fork contrary to their will. If eventually the child
chain becomes more popular than the parent blockchain, the parent blockchain can
eventually be sunset, completing the upgrade process. This technique can also be
used to \emph{abandon ship} in case a parent blockchain is found to have a
long-term security issue which cannot be immediately exploited, such as a
suspicion for a cryptographically secure hash weakness. If enough time is
available to create a sidechain and have users move all their assets to it, this
method allows securely moving away from a suspected insecure system. In this
case, the two-way peg could be carefully constructed to have a time limited
after which no more funds can further be transferred from the parent blockchain
into the child blockchain, effectively severing the link between the two and
making the child blockchain independent.

% TODO: evaluation using Giorgos' work
