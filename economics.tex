% \section{Sidechain economics}
%
% \noindent
% \textbf{Macroeconomic firewalling. }
% It remained implicit in our previous constructions that assets transferred to
% and from a sidechain are \emph{macroeconomically firewalled}. In our two-way
% pegged construction, one of the two blockchains, the \emph{main chain}, was
% functioning as the blockchain in which the deposits were made in its
% \emph{native currency}. The main chain's miners and full nodes maintain a
% certain \emph{macroeconomic policy} which includes the current money supply and
% its emission rate. We note that these policies cannot be violated in case a
% contract enabled moving the asset back and forth between a remote chain. This is
% first of all necessary because the miners enforcing the policy are not aware of
% the nature of the smart contract, and so it would be impossible for the policy
% to be altered by it. However, it is worth observing the implications of this.
% Consider a remote chain in which a \emph{catastrophic failure} occurs; for
% example, its security assumption, such as the honest majority, are violated. In
% that case, the main chain's policy is protected. The reason is that the amount
% of money that can be withdrawn from the smart contract residing within the main
% chain is limited to the amount that was deposited into it (it does not hold any
% money to pay additional dues). While, in case of such failure, it is possible
% that an adversary is able to withdraw these amounts, the participants who did
% not participate in the sidechain are unaffected. This has already been observed
% in previous work~\cite{sidechains}.
%
% \noindent
% \textbf{Merged mining. }
% We have described the technical means by which two blockchains can communicate
% in a decentralized manner. Many of the applications, such as atomic swaps and
% remote ICOs, can immediately benefit from this construction, as it pertains to
% two blockchains that remain independently secure and are stand-alone. Other
% applications, such as upgradability and scalability, envision the sidechains to
% evolve as \emph{children} blockchain of the parent blockchain and not to be
% independent. In such a setting, the child chain does not have its own asset, but
% instead borrows the asset native to its parent chain. One of the central
% questions in this construction pertains to the incentives of the child chain
% miners. If mining child chain blocks does not issue a reward, participants are
% not incentivized to mine there, noting that proof-of-work is costly and that
% usage of the child chain does not require the user herself to mine on it.
%
% One approach to solve this problem which has been proposed~\cite{sidechains} is
% to allow parent chain miners to \emph{merge mine}~\cite{blockstack} on the child
% chain. % TODO
%
