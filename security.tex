\section{Security analysis}

We now formalize our protocol and provide a cryptographic analysis of its
security. As NIPoPoWs security is modelled in the Bitcoin
Backbone Protocol~\cite{EC:GarKiaLeo15}, we work in the same model (and note
that the same mathematical model also captures Ethereum). We assume that the
standard results of the backbone protocol are attained, namely blockchain
persistence and liveness. Persistence and liveness can be proved to hold with
overwhelming probability under the honest mining majority assumption. For the
details of that result, consult the Bitcoin Backbone
paper~\cite{EC:GarKiaLeo15}.

We adopt the sidechains security definition from related work on proof-of-stake
sidechains~\cite{pos-sidechains}.

We will show that proving, to the maintainers of a chain $\mathcal{G}_2$, that
an event $e$ took place in chain $\mathcal{G}_1$ without it actually happening,
can only occur if the underlying NIPoPoWs protocol is insecure. Therefore, our
proof strategy follows the standard form of a cryptographic computational
reduction. In our assumptions, we will make use of the \emph{persistence} and
\emph{liveness} of $\mathcal{G}_2$ and only the \emph{persistence} of
$\mathcal{G}_1$.

\begin{theorem}
  Assume a secure NIPoPoWs construction. Then, under the honest majority
  assumption for both $\mathcal{G}_1$ and $\mathcal{G}_2$, for all PPT
  adversaries $\mathcal{A}$ and for all environments $\mathcal{Z}$, the
  proof-of-work sidechains construction between $\mathcal{G}_1$ and
  $\mathcal{G}_2$ with contestation period $2k$ is secure, except with
  negligible probability in $k$.
\end{theorem}
\begin{proof}
  Let $\mathcal{A}$ be an arbitrary PPT adversary against the proof-of-work
  sidechains construction and $\mathcal{Z}$ be an arbitrary environment. We will
  construct an adversary $\mathcal{A}^*$ against NIPoPoWs and an environment
  $\mathcal{Z}^*$ in which it will operate.

  Suppose, without loss of generality, that $\mathcal{A}$ can break the security
  of proof-of-work sidechains during a cross-chain transfer from $\mathcal{G}_1$
  to $\mathcal{G}_2$. (Because the construction is symmetric, if the adversary
  is not able to do that, then they will be able to break the security of a
  cross-chain transfer from $\mathcal{G}_2$ to $\mathcal{G}_1$ and the proof
  follows in the same manner.)

  Note that $\mathcal{A}$ works in an environment with two blockchains,
  $\mathcal{G}_1$ and $\mathcal{G}_2$, while $\mathcal{A}^*$ must work in the
  environment of one blockchain, namely $\mathcal{G}_1$.

  % TODO: Make Z* swallow the environment of Z
  % TODO: Mention that the NIPoPoWs that we use have adaptive predicates
  % LIV_1 -> LIV_2
  $\mathcal{A}^*$ works as follows. First, it simulates the execution of the
  blockchain civilization $\mathcal{G}_2$. That is, it creates a new random
  oracle for $\mathcal{G}_2$ which is independent of its external random oracle
  used with $\mathcal{G}_1$. For any random oracle queries of $\mathcal{A}$
  pertaining to $\mathcal{G}_1$, $\mathcal{A}^*$ forwards the queries to its
  external random oracle. For random oracles queries of $\mathcal{A}$ pertaining
  to $\mathcal{G}_2$, $\mathcal{A}^*$ answers its queries with its simulated
  and independent random oracle. Because $\mathcal{A}$ is subject to honest
  majority limitations in both $\mathcal{G}_1$ and $\mathcal{G}_2$, it follows
  that $\mathcal{A}^*$ will respect honest majority with regards to its external
  random oracle. For any environment instructions requested by $\mathcal{Z}$
  pertaining to $\mathcal{G}_1$ (namely, the creation of new parties), the
  instructions are mirrored by $\mathcal{Z}^*$. Intructions of $\mathcal{Z}$
  pertaining to $\mathcal{G}_2$ are simulated by $\mathcal{A}^*$. All diffusions
  of blocks in $\mathcal{G}_1$ by $\mathcal{A}$ are also diffused by
  $\mathcal{A}^*$, while diffusions in $\mathcal{G}_2$ by $\mathcal{A}$ are held
  private.

  $\mathcal{A}^*$ monitors the chains adopted by honest parties and for every
  round $r$ observes the state of all honest parties. $\mathcal{A}^*$ looks for
  a round $r$, an event $e$, a $\mathcal{G}_1$ maintainer $p_1$ and a
  $\mathcal{G}_2$ maintainer $p_2$ for which the following properties hold:

  \begin{enumerate}
    \item $p_1$ has not included $e$ in their state
    \item $p_2$ has included $e$ in their \textsf{finalized-events} state
  \end{enumerate}

  Because of the construction of $p_2$, \textsf{finalized-events} can contain
  $e$ only if an issuance of \textsf{submit-event-proof} is included at least
  $2k$ blocks deep and contains the respective NIPoPoW $\pi$ stored in
  $\textsf{events}[e].\textsf{proof}$. $\mathcal{A}^*$ now returns the proof
  $\pi$.

  We will now analyze the probability of success of $\mathcal{A}$. Consider
  the following (probabilistic) events:

  \begin{enumerate}
    \item $\textsc{SC-Brk}$ that $\mathcal{A}$ is successful
    \item $\textsc{Cert-Brk}$ that $\mathcal{A}^*$ is successful
    \item $\textsc{Per}_1$ that persistence is maintained in $\mathcal{G}_1$
    \item $\textsc{Per}_2$ that persistence is maintained in $\mathcal{G}_2$
    \item $\textsc{Live}_2$ that liveness is maintained in $\mathcal{G}_2$
    \item $\textsc{BC}$ the union of $\textsc{Per}_1 \land \textsc{Per}_2 \land \textsc{Live}_2$
  \end{enumerate}

  From total probability we obtain:

  \[
  \Pr[\textsc{SC-Brk}] = \Pr[\textsc{SC-Brk}|\textsc{BC}]\Pr[\textsc{BC}]
                       + \Pr[\textsc{SC-Brk}|\lnot \textsc{BC}]\Pr[\lnot \textsc{BC}]
  \]

  From the honest majority assumption of $\mathcal{G}_1$, we deduce that
  $\Pr[\lnot \textsc{Per}_1]$ and $\Pr[\lnot \textsc{Live}_1]$ are negligible,
  and similarly from the honest majority assumption of $\mathcal{G}_2$ we deduce
  that $\Pr[\lnot \textsc{Per}_2]$ is negligible, therefore
  $\Pr[\lnot \textsc{BC}]$ is negligible. It now suffices to show that
  $\Pr[\textsc{SC-Brk}|\textsc{BC}]\Pr[\textsc{BC}]$ is negligible.

  Suppose that $\textsc{SC-Brk}$ occurs. It follows that a (blockchain) event
  $e$ must have been adopted by $p_2$ with some NIPoPoW $\pi$, but not by $p_1$,
  as detailed above. Suppose now that $\textsc{BC}$ occurs.

  Because of the \emph{persistence} of $\mathcal{G}_2$, when $\pi$ was burried
  under $k$ blocks in the adopted chain of $p_2$, all honest parties in
  $\mathcal{G}_2$ must have seen $\pi$ (this warrants the oldest $k$ of the $2k$
  blocks in the contestation period). Because of the \emph{liveness} of
  $\mathcal{G}_2$, at least one honest block must have been included in the last
  $k$ blocks after $\pi$ had been received by all honest parties (this warrants
  the latest $k$ of the $2k$ blocks in the contestation period).

  Because of the \emph{persistence} of $\mathcal{G}_1$, if $e$ is not included
  in the state of $p_1$ at round $r$, then therefore it cannot have been
  included in the state of any $\mathcal{G}_1$ party during round $r - \eta k$.
  It follows that an honest party will attempt and succeed in generating a
  $\mathcal{G}_2$ block containing a contesting proof $\pi^*$ attesting to the
  fraudulence of event $e$ by invoking
  $\textsf{submit-contesting-proof}(\pi^*, e)$ and this block will be adopted by
  $p_2$. As $p_2$ has finalized $e$, then therefore it must be such that
  $\textsf{verify}^{e, \mathcal{G}}_{k,m}(\{\pi, \pi^*\})$, and therefore
  \textsc{Cert-Brk} has occurred.

  Putting the above together, we obtain that:

  \[
  \Pr[\textsc{Cert-Brk}] \geq \Pr[\textsc{SC-Brk}|\textsc{BC}]\Pr[\textsc{BC}]
  \]

  From the NIPoPoW security assumption, we have that $\Pr[\textsc{Cert-Brk}]$ is
  negligible. Therefore, $\Pr[\textsc{SC-Brk}]$ is negligible.
\end{proof}
